\documentclass[letterpaper,10pt]{article}
\usepackage[utf8]{inputenc}
\usepackage[margin=1in]{geometry}
\usepackage{graphicx, verbatim, hyperref, color, soul, multicol}

\newcommand{\code}[1]{
\subsection{\texttt{#1}}
\label{sec:#1}
\texttt{\verbatiminput{#1}}
}

\newcommand{\ig}[1]{
\begin{center}
    \includegraphics[width=3in]{{{#1}}}
\end{center}
}

\title{NEU 501a -- HW3}
\author{Tom Bertalan}

\begin{document}

\maketitle

\section{Binocular Rivalry}
Code is in \S\ref{sec:serviceCode}.


\subsection{Single I/F Neuron}

\subsubsection{Spiking Threshold}
See \texttt{part1A} in \S\ref{sec:hw3-1.py}.
 
For plotting purposes, voltages above $V_\mathrm{thresh}=-50\,[mV]$ are here replaced
with sudden $+10\,[mV]$ spikes. Bifurcation to a spiking regime occurs at $I=0.3\,[nA]$.

\ig{501a-hw3-part1A}

\subsubsection{Ramped Input Current}
\begin{multicols}{2}
See \texttt{part1B} in \S\ref{sec:hw3-1.py}.

$$C\frac{dV}{dt} = I - g_L(V(t) - V_l)$$

Solution:
$$ V(t) = c_1 e^{-g_lt/C} + I/g + V_l $$

Let $V(t=0) = V_l$, so $c_1=-I/g$.
$$V(t) = I(1 - e^{-g_lt/C}) / g + V_l$$

Solve for $t$:
$$t(V) = -c/g \ln{\left(1-\frac{g_l}{I}(V(t) - V_l)\right)}$$

So, the expected frequency is
$$f = 1 / [t_\mathrm{ref} + t(V_\mathrm{thresh})]
    = 1 / \left[t_\mathrm{ref} - C/g_l \ln{\left(1 - g/I(V_\mathrm{thr} - V_l)\right)}\right]$$
Where $t_\mathrm{ref}$ is the refractory time of the neuron.

\ig{part1B}

\end{multicols}

\newpage
\subsubsection{Gaussian-Random Input Current}
See \texttt{part1C\_ramp} in \S\ref{sec:hw3-1.py}.

The average shape of the $f/I$ curve doesn't seem to be affected, although a
spread is imposed.

\includegraphics[width=\textwidth]{{{part1C}}}

\hl{I still need to look at the effect on inter-spike times with constant $I$ and
varying $I_\mathrm{rand}$.
I'll add this non-ramping version on Saturday. But it should just be a slice of the
ramping version.
}

\newpage
\subsection{Two-I/F-Neuron Network}
See \S\ref{sec:twoNet.py} for the code to generate the basic two-neuron network.

Averaged across 20 trials the ratio of neuron one's frequency to neuron two's
frequency appears to be a smooth function of their ratio of input currents.

With no noise ($I_\mathrm{rand}=0$), the an unstable limit cycle is revealed
in which the two neurons oscillate in synchrony. However, the smallest amount  
of noise ($I_\mathrm{rand}=0.8$) breaks this symmetry, and the network appears
to fall into a stable limit cycle of antisymmetric firing. This degenerates as
$I_\mathrm{rand}$ is further increased.

\begin{multicols}{2}

\includegraphics[width=3in]{{{driveRatio-20runs}}}

\includegraphics[width=3in]{{{part2_timecourse-Irand0.80}}}

\includegraphics[width=3in]{{{part2_timecourse-Irand0.00}}}

\includegraphics[width=3in]{{{part2_timecourse-Irand3.00}}}

\end{multicols}

\newpage
\subsection{Poisson-random Synaptic Increments}
With clean $1\,[nA]$ input currents to both neurons, but with synaptic increases
per spike being $0.2 s(m)\,[mS/msec]$, where $s$ is Poisson-distributed
with mean $m$, sufficiently large values of $m$ lead to bistability.

Here, $m$ was set to $40$. At $t=1000\,[msec]$, The lower of the two
synaptic conductances was abruptly increased to $0.7\,[\mu S]$ (analogous to a
sudden synaptic increment), causing the system to switch into its other
bistable state.
\ig{part3}

\hl{I should try uneven $I_\mathrm{DCi}$ (uneven saliency), and behavior
as a function of $m$.}

\newpage
\section{Bistability in Attractor Spaces}
\hl{For future release.}

\section{Service Code}
\label{sec:serviceCode}
Code is also online at
\href{https://github.com/tsbertalan/501A-hw3}{github.com/tsbertalan/501A-hw3}.

\code{integrateAndFire.py}

\newpage
\code{twoNet.py}

\newpage
\section{User Code}
\label{sec:userCode}
\code{hw3-1.py}


\end{document}
